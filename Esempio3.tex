\documentclass[11pt]{book}
 \usepackage[utf8]{inputenc}
 \usepackage[T1]{fontenc}
 \usepackage[italian]{babel}
 \usepackage[sc]{mathpazo}
 \linespread{1.05}

\usepackage[paperwidth=120mm,paperheight=210mm,top=12mm,bottom=25mm,outer=20mm,inner=13mm]{geometry}
\usepackage[]{matrita}
\usepackage{indentfirst}
\usepackage{xcoffins}
\usepackage{microtype}
\usepackage{lipsum}
\usepackage{xcolor}
\usepackage{xstring}
\usepackage{expl3}
\usepackage{textcase}
\usepackage{graphicx}
\usepackage{lettrine}
\usepackage{fancyhdr}
\pagestyle{fancy}
\fancyhead{} % clear all header fields
\fancyfoot{} % clear all footer fields
\renewcommand{\headrulewidth}{0pt}
\renewcommand{\footrulewidth}{0pt}

\renewcommand{\respfont}{\bfseries}
\setlength{\parindent}{0pt}
\definecolor{respcolor}{rgb}{1,0,0}
\definecolor{etgray}{gray}{0.8}
\setlength{\afterpoemtitleskip}{2ex plus 0ex minus 1ex}
\setlength{\beforepoemtitleskip}{2.5ex plus 1ex minus 2ex}
\setlength{\leftmargini}{0em}
\setlength{\titleindent}{0em}
%\setlength{\parskip}{0.3\baselineskip}
\renewcommand{\poemtitlefont}{\normalfont\large\bfseries}
\definecolor{crosscolor}{rgb}{1,0,0}
\renewcommand{\intestfont}[1]{{\Large\scshape\textcolor{red}{#1}}}
\renewcommand{\nomelibrofont}[1]{{\bfseries#1}}
\ExplSyntaxOn
\NewCoffin\InitialCoffin
\NewCoffin\RestCoffin
\NewCoffin\LineCoffin
\newlength{\InitKernCorr}
\tl_new:N \Part_Title_tl
\tl_new:N \Rest_of_Title_tl
\tl_set:Nn \First_Title_tl {\tl_head:N \Part_Title_tl}
\tl_set:Nn \Rest_of_Title_tl {\tl_tail:N \Part_Title_tl}
\RenewDocumentCommand {\momento}{O{0em}m}{
  \tl_set:Nn \Part_Title_tl {#2}
  \setlength{\InitKernCorr}{#1}
  \SetHorizontalCoffin\InitialCoffin{
    \normalfont\scalebox{2}{\Large\textcolor{red}{\First_Title_tl}\hspace{\InitKernCorr}}
  }
  \SetHorizontalCoffin\RestCoffin{
    \normalfont\Large\textcolor{red}{\MakeTextUppercase	\Rest_of_Title_tl}
  }
  \SetHorizontalCoffin\LineCoffin{
    \textcolor{black}{\rule[-1.5pt]{\dimexpr\textwidth-\CoffinWidth\InitialCoffin\relax}{0.6pt}}
  }
  \JoinCoffins\LineCoffin[l,t]\RestCoffin[l,b]
  \JoinCoffins\LineCoffin[l,b]\InitialCoffin[r,b]
  \par\vspace*{5\baselineskip}\noindent\TypesetCoffin\LineCoffin (0mm, 0mm)\vspace{3\baselineskip}
}
\ExplSyntaxOff
\newcommand{\sottomomento}[1]{{\intestfont{#1}}\par\medskip}
\begin{document}
\momento[0.05em]{Riti di introduzione}
\segnocroce

\newpage
\sottomomento{Memoria del Battesimo}
\introduzione

\membatt
\newpage
\sottomomento{Inno di Lode}
\gloria
\sottomomento{Colletta}
\colletta
\newpage
\momento{Liturgia della parola}
\begin{lettura}[Prima]{Dagli Atti degli Apostoli}{At\,2,\,42--47}
\lettrine[lines=3]{E}{rano} perseveranti nell'insegnamento degli apostoli e nella comunione, nello spezzare il pane e nelle preghiere. Un senso di timore era in tutti, e prodigi e segni avvenivano per opera degli apostoli. Tutti i credenti stavano insieme e avevano ogni cosa in comune; vendevano le loro proprietà e sostanze e le dividevano con tutti, secondo il bisogno di ciascuno. Ogni giorno erano perseveranti insieme nel tempio e, spezzando il pane nelle case, prendevano cibo con letizia e semplicità di cuore, lodando Dio e godendo il favore di tutto il popolo. Intanto il Signore ogni giorno aggiungeva alla comunità quelli che erano salvati.
\end{lettura}

\newpage%************************
\renewcommand{\versettosalmo}{Lodiamo insieme il nome del Signore}
\noindent\intestfont{Salmo responsoriale}\hfil {\small\itshape\textcolor{red}{dal Salmo 148}}

\medskip\nobreak
\noindent\rispostasalmo

\nobreak
\begin{verse}
Lodate il Signore dai cieli,\\
lodatelo nell'alto dei cieli.\\
Lodatelo, voi tutti, suoi angeli,\\
lodatelo, voi tutte, sue schiere.\\
\rispostasalmo

Lodatelo, sole e luna,\\
lodatelo, voi tutte, fulgide stelle.\\
Lodatelo, cieli dei cieli,\\
voi acque al di sopra dei cieli.\\
\rispostasalmo

Monti e voi tutte, colline,\\
alberi da frutto e tutti voi, cedri,\\
voi fiere e tutte le bestie,\\
rettili e uccelli alati.\\
\rispostasalmo


I re della terra e i popoli tutti,\\
i governanti e i giudici della terra,\\
i giovani e le fanciulle,\\
i vecchi insieme ai bambini.\\
\rispostasalmo
\end{verse}
\newpage%************************
\begin{lettura}[Seconda]{Dalla prima lettera ai Corinzi}{1Cor\,13,\,1--13}
\lettrine[lines=3]{S}{e parlassi} le lingue degli uomini e degli angeli, ma non avessi la carità, sarei come bronzo che rimbomba o come cimbalo che strepita.

E se avessi il dono della profezia, se conoscessi tutti i misteri e avessi tutta la conoscenza, se possedessi tanta fede da trasportare le montagne, ma non avessi la carità, non sarei nulla.

E se anche dessi in cibo tutti i miei beni e consegnassi il mio corpo per averne vanto, ma non avessi la carità, a nulla mi servirebbe.

La carità è magnanima, benevola è la carità; non è invidiosa, non si vanta, non si gonfia d'orgoglio, non manca di rispetto, non cerca il proprio interesse, non si adira, non tiene conto del male ricevuto, non gode dell'ingiustizia ma si rallegra della verità. Tutto scusa, tutto crede, tutto spera, tutto sopporta.

La carità non avrà mai fine. Le profezie scompariranno, il dono delle lingue cesserà e la conoscenza svanirà. Infatti, in modo imperfetto noi conosciamo e in modo imperfetto profetizziamo. Ma quando verrà ciò che è perfetto, quello che è imperfetto scomparirà. Quand'ero bambino, parlavo da bambino, pensavo da bambino, ragionavo da bambino. Divenuto uomo, ho eliminato ciò che è da bambino.

Adesso noi vediamo in modo confuso, come in uno specchio; allora invece vedremo faccia a faccia. Adesso conosco in modo imperfetto, ma allora conoscerò perfettamente, come anch'io sono conosciuto. 

Ora dunque rimangono queste tre cose: la fede, la speranza e la carità. Ma la più grande di tutte è la carità!
\end{lettura}

\newpage%************************
\sottomomento{Canto al Vangelo}

\begin{vangelo}{Matteo}{Mt\,5,\,1--12}
\lettrine[nindent=-1pt,slope=-0.4em,lines=3]{V}{\,edendo} le folle, Gesù salì sul monte: si pose a sedere e si avvicinarono a lui i suoi discepoli. Si mise a parlare e insegnava loro dicendo:

<<Beati i poveri in spirito, perché di essi è il regno dei cieli.
Beati quelli che sono nel pianto, perché saranno consolati.
Beati i miti, perché avranno in eredità la terra.
Beati quelli che hanno fame e sete della giustizia, perché saranno saziati.
Beati i misericordiosi, perché troveranno misericordia.
Beati i puri di cuore, perché vedranno Dio.
Beati gli operatori di pace, perché saranno chiamati figli di Dio.
Beati i perseguitati per la giustizia, perché di essi è il regno dei cieli.

Beati voi quando vi insulteranno, vi perseguiteranno e, mentendo, diranno ogni sorta di male contro di voi per causa mia. Rallegratevi ed esultate, perché grande è la vostra ricompensa nei cieli. Così infatti perseguitarono i profeti che furono prima di voi.>>
\end{vangelo}
\newpage%************************
\momento{Liturgia del Matrimonio}
\sottomomento{Interrogazioni prima del Consenso}
\matrintro
\medskip

\matrpre
\medskip
\newpage%************************
\sottomomento{Manifestazione del Consenso}
\consintro
\medskip

\promesse
\medskip

\sottomomento{Accoglienza del Consenso}
\preghpost

\newpage%************************
\sottomomento{Benedizione e consegna degli anelli}
\benedizioneanelli

\medskip

\consegnanello


\newpage%************************
\sottomomento{Preghiere dei fedeli\\ e invocazione dei Santi}
\introfedeli
\preghierefedeli
\pagebreak

\introlitanie

\litanie
\newpage%************************
\sottomomento{Professione di fede}
\credo
\newpage%************************
\momento{Liturgia Eucaristica}
\sottomomento{Orazione sulle Offerte}
Accogli, Signore, i doni e le preghiere che Ti presentiamo
per \sposa{} e \sposo, uniti nel vincolo
santo: questo mistero, che esprime la pienezza della
tua carità, custodisca per sempre il loro amore. \par\nobreak
Per Cristo nostro Signore.
\rispostatutti{Amen}
\newpage%************************
\sottomomento{Prefazio del Matrimonio I}
\prefazio

\medskip
{\bfseries\santosanto}
\newpage%************************
\sottomomento{Preghiera Eucaristica III}
\pregheucar
\misterofede
\orazionieucar
\sottomomento{Benedizione degli sposi}
\benedizionesposi[1]


\sottomomento{Orazione dopo la Comunione}
\preghierecomunione
\newpage%************************
\momento{Riti di Conlusione}
\articolilegge
\newpage%************************
\momento{Benedizione solenne}
\benedizionefinale
\congedo


\end{document}
